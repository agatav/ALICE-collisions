\documentclass[11pt,a4paper,twoside]{article}
\usepackage[includeheadfoot, left=3cm, right=2cm, top=2.5cm, bottom=2.5cm, headheight=19.3pt]{geometry}

\usepackage[russian,english,main=polish]{babel}
\usepackage[T1]{fontenc}
\usepackage[utf8]{inputenc}
\usepackage[scaled]{helvet}
\renewcommand\familydefault{\sfdefault} 

\let\lll=\relax
\usepackage{amsmath}
\usepackage{amsfonts}
\usepackage{amssymb}
\usepackage{bm}
\usepackage{url}
\usepackage{pdfpages}
\usepackage{footnote}
\usepackage[bookmarks, pagebackref]{hyperref}
\usepackage[none]{hyphenat}
\usepackage{graphicx}
\usepackage{color}
\usepackage{array}
\usepackage{etoolbox, fancyhdr, xcolor}
\usepackage[hang, flushmargin]{footmisc}
\usepackage{setspace}
\usepackage{ragged2e}
\usepackage{MnSymbol}
\usepackage[nottoc]{tocbibind}
\usepackage{titlesec}
\urlstyle{rm}
\usepackage[ampersand]{easylist}
\usepackage{enumitem}
\setlist[itemize]{noitemsep, topsep=0pt, leftmargin=*, itemindent=-1cm}
\usepackage{epsfig}
\usepackage{float}
\usepackage[font={up, footnotesize}, labelfont=bf, singlelinecheck=off, format=hang]{caption}
\usepackage{caption}
\usepackage{subcaption}
\usepackage{listings}
\lstset{language=C++, rulecolor=\color{sapphire}, framerule=1.5pt} % you can change language of your code
\usepackage{colortbl}
\usepackage{tabu}
\usepackage{makecell}
\usepackage{boldline}
\setlength{\arrayrulewidth}{1pt}
\captionsetup[table]{name=Tabela}
\usepackage{multirow}

\usepackage{lipsum} % for testing purposes

\definecolor{sapphire}{RGB}{120,150,207}
\definecolor{grafit}{RGB}{60,60,60}

% pdf output setup
\hypersetup{
	unicode,
	pdftoolbar,
	pdfmenubar,
	pdffitwindow,
	pdfstartview = {FitH},
	pdftitle = {Praca Inżynierska}, 
	pdfauthor = {Ahata Valiukevich},
	pdfnewwindow,
	colorlinks,
	linktoc = page,
	% use color sapphire or grafit
	linkcolor = sapphire,
	citecolor = sapphire,
	filecolor = sapphire,
	urlcolor = black
}


\newcommand{\itemi}[1][sapphire]{\item[\color{#1} $\filledsquare$]}
\newcommand{\itemii}[1][sapphire]{\item[\color{#1} $\square$]}
\newcommand{\itemiii}[1][sapphire]{\item[\color{#1} $\bullet$]}

\ListProperties(Hide=100, Progressive=1cm, Style=\color{sapphire}, Style**=\color{black}, Style*=$\square$ ,Style2*=$\bullet$)


\newcommand{\headrulecolor}[1]{\patchcmd{\headrule}{\hrule}{\color{#1}\hrule}{}{}}
\newcommand{\footrulecolor}[1]{\patchcmd{\footrule}{\hrule}{\color{#1}\hrule}{}{}}


\newcommand{\infostyle}[1]
{
	\fancyhf{}
	\fancyhead[LO]{\Large{\textbf{#1}}}

	\renewcommand{\headrulewidth}{1.5pt}
	\headrulecolor{sapphire}
	\justify
	\pagestyle{fancy}
}


\newcommand{\thesisstyle}
{
	\fancyhf{}
	\fancyhead[RO,LE]{\Large{\textbf{\rightmark}}}
	\fancyfoot[LE,RO]{\thepage}

	\renewcommand{\headrulewidth}{1.5pt}
	\renewcommand{\footrulewidth}{1.5pt}
	\headrulecolor{sapphire}
	\footrulecolor{sapphire}
	\justify
	\pagestyle{fancy}
}


\newcommand{\fancyfootnotetext}[2]
{
	\fancypagestyle{footnotes}
	{
    	\fancyfoot[LO,RE]{\parbox{15cm}{\footnotemark[#1]\footnotesize #2}}
	}
	\thispagestyle{footnotes}
}


\newcommand{\fancyfootnotetexts}[4]
{
	\fancypagestyle{footnotes}
	{
    	\fancyfoot[LO,RE]{\parbox{15cm}{\footnotemark[#1]\footnotesize #2 \\ \footnotemark[#3]\footnotesize #4}}
	}
	\thispagestyle{footnotes}
}


\newcommand{\fancyfootnotetextss}[6]
{
	\fancypagestyle{footnotes}
	{
    	\fancyfoot[LO,RE]{\parbox{15cm}{\footnotemark[#1]\footnotesize #2 \\ \footnotemark[#3]\footnotesize #4 \\ \footnotemark[#5]\footnotesize #6 \\}}
	}
	\thispagestyle{footnotes}
}


\titlespacing\section{0pt}{15pt}{10pt} 
\titlespacing\subsection{0pt}{15pt}{10pt} 
\titlespacing\subsubsection{0pt}{15pt}{10pt} 

\setstretch{1.15}
\setlength{\parindent}{0.5cm} 
\setlength{\parskip}{0cm} 

\frenchspacing
\sloppy

\author{} 
\title{} 
\date{}
\graphicspath{{images/}}

\begin{document}

\includepdf[pages={-}]{preamble/begining.pdf}
\tableofcontents
\thispagestyle{empty}
\thesisstyle
\newpage 

\section[Wstęp]{Wstęp}
W przeciągu ostatnich kilku lat nauka zrobiła duży postęp w kierunku rozwoju fizyki dużych energii. Osiągnięcie to składa się z wyników dużej liczby badań oraz eksperymentów prowadzonych w różnych naukowo-badawczych instytucjach. Równolegle z naukami fizycznymi rozwiajała się również branża informatyczna, a w szczególności grafika komputerowa. Obecnie jest obserwowany coraz silniejszy trend związany z rozwojem technik stereoskopowych. Przy połączeniu badań fizycznych i grafiki komputerowej powstaje możliwość dogłębnego przestudiowania zjawisk zachodzących podczas rożnych eksperymentów.


\subsection{Cele i zakres pracy}
Pierwszym celem niniejszej pracy dyplomowej jest stereoskopwa wizualizacja detektora oraz zdarzeń ciężkich ionów w eksperymencie ALICE. Kolejnym istotnym celem pracy jest połączenie wyżej wymienionych wizualizacji do jednego spójnego obrazu. 
\paragraph{}
Dla realizacji celi podstawowych niezbędne jest zapoznanie się z technologią stereoskopii i aktualnie dostępnymi rozwiązaniami,przestudiowanie podstaw działania eksperymentu ALICE oraz środowiska informatycznego w CERN. \paragraph{}
Celem pośrednim danej pracy jest zapoznanie się z możliwościami biblioteki OpenGL, a także innymi uwarunkowaniami ważymi w aspekcie projektowanej aplikacji.

\newpage
\section[Część analityczna]{Część analityczna}

\subsection{Stereoskopia}
Stereoskopią nazywa się technikę służącą do tworzenia iluzji głębi obrazu. Ta metoda nie zapewnia prawdziwie trójwymiarowych widoków, ale zapewnia efekt trójwymiarowości. Sceny wydają się mieć głębię, dzięki temu że dla każdego oka obserwatora jest prezentowany inny widok. 
\subsubsection{Historia} 
Historia stereoskopii liczy już ponad 180 lat. Za odkrywcę tej techniki uważa się Charlesa Wheatstone'a, który jako pierwszy zaobserwował zjawisko widzenia stereoskopowego, opisał go w artykule Philosophical Transactions\footnote{Wheatstone, Charles. “Contributions to the Physiology of Vision.–Part the First. On some remarkable, and hitherto unobserved, Phenomena of Binocular Vision.” Philosophical Transactions of the Royal Society. Vol. 128 (1838) pp. 371-394. http://www.stereoscopy.com/library/wheatstone-paper1838.html (2009-05-27).} oraz skonstruował urządzenie potwierdzające słuszność jego teorii\footnote{Brewster, David. The Stereoscope: its history, theory, and construction, with its application to the fine and useful arts and to education. London, 1856}. 
\paragraph{} Wheatstone wyjaśnia, że gdy patrzy się na obiekty znajdujące się w dużej odległości, to nie ma znaczenia, którym okiem to się robi, ponieważ wygląd przedmiotów jest dokładnie ten sam dla obu oczu. Ale jeśli obiekt znajduje się znacznie bliżej, to każde oko widzi przedmiot pod nieco innym kątem. Wykorzystując tę tezę, autor skonstruował urządzenie, które umożliwiło przedstawienie delikatnie różniących się obrazków prawemu i lewemu oku z osobna. W efekcie powstawała iluzja trójwymiarowości obrazu.
  
\subsubsection{Popularne techniki stereoskopowe}
Istnieje kilka różnych technik osiągnięcia efektu stereoskopowego. Dla każdej techniki przeglądania niezbędny jest materiał stereoskopowy dostosowany do konkretnego celu. Poniżej są przedstwione krótkie opisy kilku popularnych technik. Niektóre z opisanych technik są odpowiednie tylko dla jednego widza, podczas gdy istnieją różne typy okularów 3D, które sprawdzają się dla większej grupy odbiorców.
\begin{itemize}
\itemi Najprostszą metodą doświadczenia widzenia stereoskopowego jest umieszczenie dwuch oddzielnych obrazów obok siebie. Perspektywa jednego z obrazów reprezentuje lewe oko, odpowiednio druga perspektywa jest dla oka prawego. Obrazy te można zazwyczaj oglądać bez użycia dodatkowych narzędzi. Jedyna różnica polega na sposobie ich oglądania. Jeśli obraz dla lewego oka znajduje się po lewej stronie, to poprawnie jest patrzeć trzymając oczy równolegle. Jeśli natomiast sytuacja jest odwrotna (obraz dla lewego oka jest po prawej stronie), obrazy mają być oglądane poprzez skrzyżowanie oczu. Użycie niewłaściwej metody oglądania spowoduje odwrócenie efektu stereoskopowego. Może to ujawnić się w postaci zniekształconych odległości pomiędzy obiektami na obrazie.\\
Warto zauważyć, że oglądanie takich obrazów tworzy duże obciążenie dla oczu. W celu zmniejszenia negatywnych skutków używa się stereoskopu. Dane urządzenie w znacznej mierze redukuje przemęczenie oczu, co ułatwia długie oglądanie obrazów.

\itemi Autostereogram – Similar to the images that are made for cross-eyed viewing,
the autostereogram images show a three-dimensional depth when the right
viewing technique is used to look at them. Though, as shown in Figure 2, they
can sometimes be hidden behind what seems to be a totally random mess of
colors. This “mess” is actually not so random for it is a repeated pattern that
conceals the three-dimensional objects [6]. Figure 3 shows an image which is
not hidden with a pattern which is an image of a chessboard where the pieces
are repeated horizontally. By using the parallel viewing technique, a
stereoscopic effect will be created as the adjacent pieces merge together. This
is also how the images with a random pattern work. By using a z-depth image
(where objects closer to the camera is lighter and objects further away darker)
of a 3D object or scene, shown in Figure 4, the final image is calculated and
hidden behind either random dots or a pattern [7].

\itemi Polarized 3D-glasses – Polarized stereoscopy is when two projectors project
images onto a silver or aluminum covered screen, where the orientation of the
electric field of the light from the projectors has been specified [8]. When
using this method to show stereoscopic material, there are two methods how
the light is polarized: linearly and circularly. The electric field of the linearly
polarized light is directed horizontally and vertically from the projectors, and
the viewer wears a pair of 3D-glasses which passes through light from only
one direction for each eye. If the glasses are tilted their polarization will no
longer match the polarization of the light and a bleeding effect between the
two images will occur. This is avoided when using circularly polarized light
and glasses. The light is then directed in a circular motion, either leftwards or
rightwards, which remains even if the glasses are tilted to either side.

\end{itemize}



\subsubsection{Techniki anaglifowe}

\newpage
\subsection{Biblioteka OpenGL}
OpenGL (z angielskiego Open Graphics Library) jest potężnym systemem graficznym stanowiącym niejako pomost między programistą a sprzętem komputera. Biblioteka ta została stworzona przez firmę \textit{Silicon Graphics}, jednego z potentatów na rynku grafiki komputerowej.\\Procedury OpenGL umożliwiają rendorowanie obiektów o rożnych poziomach skomplikowania zaczynając od prostego punktu geometrycznego, linii lub wypełnionego wielokąta do utworzenia najbardziej złożonej, zakrzywionej powierzchni, oświetlonej i odwzorowanej teksturą. OpenGL pozwala programistom na dostęp do prymitywów geometrycznych i obrazowych, list wyświetlania, przekształcania modelu, oświetlenia i teksturowania, antyaliasingu, mieszania i wielu innych funkcji. Wszystkie elementy stanu OpenGL - nawet treść pamięci tekstur i bufor ramki - można uzyskać za pomocą aplikacji OpenGL. Dany system obsługuje także aplikacje wizualizacji z obrazami 2D traktowanymi jako typy prymitywów, którymi można manipulować podobnie jak obiektami geometrycznymi 3D.\\Mimo że specyfikacja OpenGL definiuje konkretny ciąg przetwarzania graficznego, dostawcy platformy mają swobodę dostosowywania konkretnej implementacji OpenGL w celu osiągnięcia sprecyzowanych celów w zakresie kosztów i wydajności. Pojedyncze wywołania mogą być wykonywane na dedykowanym sprzęcie, uruchamiane jako procedury programowe w standardowym systemie CPU lub implementowane jako kombinacja zarówno dedykowanych procedur sprzętowych, jak i programowych. Ta elastyczność implementacji skutkuje ​​przyśpieszeniem renderowania, w dodatku jest powszechnie dostępna na wszystkich jednostkach od komputerów o niskich kosztach, po wysokiej klasy stacjach roboczych i superkomputerach.\\
\begin{figure}[h]
\includegraphics[width=6cm]{Hierarchy_UNIX.jpg}
\centering
\caption{Schemat ilustrujący relacje OpenGL GLU}
\end{figure}
\footnote{OpenGl podstawy wg https://www.opengl.org/about/}

\subsubsection{Shaders}
W OpenGL wszystko jest przedstawione w przestrzeni trójwymiarowej, ale na ekranie obraz widzimy listę pikseli 2D, w związku z tym duża część pracy OpenGL polega na zmianie współrzędnych 3D na piksele 2D, które by pasowały do ekranu. Cały proces transformacji jest zarządzany poprzez ciąg graficzny OpenGL. Ciąg przetwarzania może być podzielony na poszczególne kroki, gdzie na wejściu każdego kroku są wymagane dane wyjściowe poprzedniego. Każdy z tych kroków jest dobrze sprecyzowany, gdyż mają one konkretną funkcję, i może być wykonany równolegle. Większość współczesnych kart graficznych posiada setki, czasmi tysiące, małych jąder procesowych do szybkiego przetwarzania danych wejściowych. W ciągu graficznym dla przyśpieszenia przetwarzania uruchamiane są nieduże programy w GPU dla każdego kroku w ciągu. Wspomniane nieduże programy są nazywane \textit{\textbf{shaderami}}.\\
Shadery są pisane w języku w dużym stopniu podobnym do C - GLSL. GLSL jest dostosowany do wykorzystania w grafice, ponieważ zawiera przydatne funkcje skierowane na manipulacje wektorami i macierzami. Shadery zawsze zaczynają się od deklaracji wersji, następnie deklarowane są listy zmiennych wejściowych i wyjściowych, uniformy oraz ich główne funkcje. 
\paragraph*{Vertex shader.}
Pierwszym krokiem ciągu graficznego jest vertex shader, który jako daną wejściową przyjmuje jeden wierzchołek. Głównym celem vertex shadera jest transformacja współrzędnych 3D do innych współrzędnych 3D. Vertex shader również pozwala robić podstawowe przetwarzanie atrybutów wierzchołka.
\paragraph*{Primitive assembly.}
Primitive assembly jest krokiem, który za dane wejściowe przyjmuje wszystkie wierzchołki (lub jeden, jeśli wybrana jest flaga GL\_POINTS ) z vertex shadera. Na tym etapie przetwarzania kształtowane są prymitywy, wszystkie wierzchołki są grupowane do zadanego kształtu.
\paragraph*{Geometry shader}
Wyjście z primitive assembly jest przekazywane do geometry shadera. Geometry shader na wejściu przyjmuje kolekcję wierzchołków, które tworzą zadany kształt oraz mogą generować inne, emitując nowe wierzchołki, tworząc nowe (lub inne) prymitywy. 
\paragraph*{Rasterization.}
Dane wyjściowe geometry shadera są podawane na wejście etapu rasterization. W tym kroku uzyskane wcześniej prymitywy mapuje się na odpowiadające im piksele na ekranie ostatecznym. W wyniku czego powstają fragmenty do przetwarzania w kolejnym kroku, ale zanim zostanie uruchomiony fragment shader, jest wykonywane przycinanie. Przyciannie pozwala odrzucić wszystkie fragmenty, które są poza zasięgiem obserwatora, ów krok pozwala znacznie zwiększyć wydajność całego ciągu.
\textit{ Fragment w OpenGL zawiera wszystkie niezbędne dane do renderowania pojedynczego piksela.}
\paragraph*{Fragment shader.}
Głównym celem fragment shadera jest obliczenie końcowego koloru piksela i jest to zazwyczaj etap, na którym występują wszystkie zaawansowane efekty OpenGL. Z reguły fragment shader posiada wszystkie dane o scenie 3D (takie jak światła, cienie, kolor światła itp.), które można użyć do obliczania końcowego koloru piksela.

\subsubsection{GLFW} 
GLFW jest to Open Source, multipatformowa biblioteka dla OpenGL, OpenGL ES. Zapewnia ona proste API do tworzenia okien, kontekstów i powierzchni, odbierania danych wejściowych i zdarzeń. GLFW jest napisana w C posiada macierzystą obsługę systemów Windows, macOS i wielu systemów uniksopodobnych, takich jak Linux i FreeBSD. 
Zalety GLFW :
\begin{enumerate}
\item Tworzy okno i cały kontekst OpenGL używając wywołania tylko 2 funkcji.
\item Obsługuje OpenGL, OpenGL ES, Vulkan i powiązane opcje, flagi oraz rozszerznia.
\item Obsługuje wiele okien, wiele monitorów, ramp o wysokiej rozdzielczości DPI i gamma.
\item Obsługuje klawiatuę, mysz, gamepad, czas i okna zdarzenia, poprzez odpytywanie lub callback.
\item Dostęp do rodzimych obiektów i opcji kompilacji dla specyficznych funkcji platformy.
\end{enumerate} 

\subsubsection{GLEW}
The OpenGL Extension Wrangler Library (GLEW) jest wieloplatformową biblioteką C / C ++. GLEW zapewnia efektywne mechanizmy wykonawcze do określania, które rozszerzenia OpenGL są obsługiwane na docelowej platformie. Funkcje jądra i rozszerzenia OpenGL są widoczne w pojedynczym pliku nagłówkowym. GLEW została przetestowany na różnych systemach operacyjnych, w tym na systemach Windows, Linux, Mac OS X, FreeBSD i Solaris.
Podczas tworzenia wizualizacji została użyta wersja statyczna GLEW, czyli glew32s.lib. \footnote{http://glew.sourceforge.net/}

\subsubsection{GLM}
OpenGL Mathematics (GLM) jest nagłówkiem tylko do biblioteki matematycznej C++ dla oprogramowania graficznego opartego na specyfikacjach języka GLSL, udostępnia klasy i funkcje zaprojektowane i zaimplementowane z użyciem tych samych konwencji nazewnictwa jak w GLSL. Mimo to GLM nie jest ograniczony tylko do cech GLSL. System rozszerzenia, oparty na konwencjach rozszerzeń GLSL, zapewnia dodatkowe możliwości: przekształcenia macierzy i kwaternionów, pakowanie danych, liczby losowe itp. \\
Ta biblioteka działa doskonale z OpenGL, ale zapewnia również współdziałanie z innymi bibliotekami i SDK innych firm. Jest dobrym kandydatem do oprogramowania renderowania (raytracing czy rasteryzacji), przetwarzania obrazu, symulacji fizycznych i dowolnego kontekstu programowania, który wymaga prostej i wygodnej biblioteki matematycznej. 
\footnote{http://glm.g-truc.net/0.9.8/index.html}



\subsection{ALICE}
A Large Ion Collider Experiment został zaprojektowany tak, aby w jak najbardziej kompletny sposób mierzyć cząstki powstałe w kolizjach, które mają miejsce w środku akceleratora, tak, aby można było zrekonstruować i zbadać ewolucję systemu w przestrzeni i czasie. Aby to zrobić, należy użyć wielu różnych detektorów, z których każdy dostarcza różne informacje. Aby zrozumieć tak złożony system, należy obserwować zjawisko z różnych punktów widzenia, przy użyciu różnych instrumentów jednocześnie. \footnote{http://aliceinfo.cern.ch/Public/en/Chapter2/Chap2Experiment-en.html}
\subsubsection{Eksperyment}
W ekstremalnych warunkach temperatury i/lub gęstości materia hadronowa "topi się" w osoczu wolnych kwarków i gluonów - tak zwanej plazmy kwarkowo-gluonowej (QGP). Aby stworzyć odpowiednie warunki w laboratorium, ciężkie jony (np. cząstki ołowiu) przyśpiesza się do niemal prędkości światła, po czym jest powodowana kolizja, zostało to zrobione w LHC w dwóch okresach w 2010 i 2011 roku. Kluczowym rozważaniem dotyczącym eksperymentu ALICE na LHC jest zdolność do badania QCD i quarków w tych ekstremalnych warunkach. Odbywa się to przy użyciu cząstek, utworzonych wewnątrz gorącej objętości podczas jej rozszerzania się i ochładzania. Cząstki te żyją wystarczająco długo, aby dotrzeć do wrażliwych warstw detektora zlokalizowanych wokół obszaru oddziaływania. Fizyka w ALICE polega na tym, żeby być w stanie zidentyfikować wszystkie z nich (tj. określić, czy są to elektrony, fotony, piony itd.), czy też określić ich ładunek. Wiąże się to w większości z różnymi sposobami oddziaływania cząstek z materią. \footnote{http://cerncourier.com/cws/article/cern/50561}
\subsection{Tracking particles}
Zespół detektorów cylindrycznych (od wewnątrz na zewnątrz: ITS Drift, ITS Strips, TPC, TRD) mierzy w wielu punktach (ponad 100 tylko dla TPC) przejście każdej cząstki przenoszącej ładunek elektryczny tak, że trajektoria jest dokładnie znana. Detektory ALICE są osadzone w polu magnetycznym (wytwarzanym przez duży czerwony magnes), wyginając w ten sposób trajektorie cząstek: z krzywizny śladów można znaleźć ich pęd. ITS jest tak precyzyjny, że cząstki, które są generowane przez rozkład innych cząstek o bardzo krótkim czasie życia można zidentyfikować, widząc, że nie pochodzą one z punktu, w którym nastąpiła interakcja ("wierzchołek" zdarzenia).
\footnote{http://aliceinfo.cern.ch/Public/en/Chapter2/Page3-subdetectors-en.html}

\newpage
\section{Projekt}
\subsection{Wymagania}
\subsubsection{funkcjonalne}
\subsubsection{niefunkcjonalne}

\subsection{Diagramy}
\subsubsection{przypadków użycia}
\subsubsection{klas}

\newpage
\subsection{Technologie}
\subsubsection{Instalowanie, linkowanie OpenGL oraz niezbędnych bibliotek}
OpenGL jest dobrze znanym standardem generowania trójwymiarowej grafiki, który jest niezwykle wydajny i posiada wiele możliwości. OpenGL jest definiowany i udostępniany przez ARB (OpenGL Architecture Review Board).
\begin{verbatim}
# apt-cache search opengl
\end{verbatim}
Istnieje wiele darmowych implementacji OpenGL dla Linux, ale potrzebna jest tylko jedna. Zainstalowana zoastała FreeGLUT, ponieważ jest ona aktualna i stanowi otwartą alternatywę dla biblioteki OpenGL Utility Toolkit (GLUT):
\begin{verbatim}
apt-get install freeglut3 freeglut3-dev libglew-dev
\end{verbatim}
Przydatne jest zainstalowanie pakietu mesa-utils, aby móc używać polecenia glxinfo:
\begin{verbatim}
# apt-get install mesa-utils
\end{verbatim}
Polecenie glxinfo wyświetla użyteczne informacje o instalacji OpenGL.
\begin{verbatim}
sudo apt-get install glew-utils
sudo apt-get install libglew-dev
sudo apt-get install libsoil-dev
\end{verbatim}
Aby utworzyć aplikację GLFW w wierszu poleceń, skorzystano z następujących opcji linkera:
\begin{verbatim}
-lglfw3 -lGL -lm -lXrandr -lXi -lX11 -lXxf86vm -lpthread
\end{verbatim}
Ostatnie trzy biblioteki są niezbędne w Ubuntu 14.04.1.

\subsection{Bezier curves}
Krzywe są zestawem nieokreślonych list punktów, które nie muszą być równe. Krzywa może być w dwuwymiarowa (krzywe płaskie) lub trójwymiarowa (przestrzeń lub krzywe przestrzeni euklidesowej). Linia jest specjalnym rodzajem krzywej, która jest prosta. Krzywa jest reprezentowana przez pewien zestaw równań, nazywany równaniem krzywej.\\
Krzywe Béziera są parametrycznymi krzywymi, które są generowane z punktów kontrolnych. Są szeroko stosowane w grafice komputerowej i innych pokrewnych branżach, ponieważ wydają się być rozsądnie gładkie na wszystkich skalach. Krzywe Béziera mają różne stopnie - liniowe krzywe, krzywa kwadratowa, krzywa sześcienna i krzywa wysokiego rzędu. \footnote{http://www.openglprojects.in/2015/12/drawing-bezier-curves-in-opengl-c.html\#gsc.tab=0}
Krzywa Beziera jest reprezentowana przez dwa punkty końcowe i dwa punkty kontrolne. Dlatego modyfikacja kształtu krzywej jest prosta. Postaci krzywej sześciennej Beziera:
\begin{equation}
B(t)=(1-t^{3})P_{0}+3(1-t)^{2}P_{1}+3(1-t)t^{2}P_{2}+t^{3}P_{3} 
\end{equation}

Stosowany wzór nazywany jest sześcienną krzywą Beziera, ponieważ mamy 2 styczne (na przykład 2 punkty w środku). P0 i P3 to dwa punkty końcowe. P1 i P2 są dwoma stycznymi punktami. Krzywa Beziera wprowadza również nową zmienną. T jest wartością pomiędzy [0-1], która w zasadzie mówi, ile punktów pośrednich chcemy między dwoma punktami końcowymi.
\begin{table}[H]
\caption{Kod źródłowy programu. Aproksymacja krzywych Beziera.}
\label{tab2}
\begin{lstlisting}[frame=single]
GLfloat bezier(float t, GLfloat P0,
                  GLfloat P1, GLfloat P2, GLfloat P3) {
  // Cubic bezier Curve
  GLfloat point = (pow((1-t), 3.0) * P0) +
    (3 * pow((1-t),2) * t * P1) +
    (3 * (1-t) * t * t * P2) +
    (pow(t, 3) * P3);
  return point;
}

void drawStuff() {
  midpoint_x = bezier(P0.x, P1.x, P2.x, P3.x, t);
  midpoint_y = bezier(P0.y, P1.y, P2.y, P3.y, t);

  drawPoint(midpoint_x, midpoint_y);
}
\end{lstlisting}
\end{table}

OpenGL ma pojęcie "\ current point "\ , zaczyna się od podania pojedynczego punktu, oblicza się następny punkt za pomocą krzywej Beziera (\ textit {\ textbf {var newMidPoint}}, a następnie podaje się nowy punkt do OpenGL. OpenGL automatycznie rysuje linię między tymi dwoma punktami, ale nowy "\ current point "\ wskazuje teraz na (\textit {\textbf {newMidPoint}}. Więc przechodzi się przez pętlę, oblicza się nowy punkt krzywej Beziera (\textit {\textbf {var secondMidPoint}}, \textit {\textbf {secondMidPoint}} do OpenGL, a OpenGL rysuje linię między (\textit {\textbf {newMidPoint -> secondMidPoint}}), powtarza się kolejne kroki itracji.

\newpage
\section[Część weryfikacyjna]{Część weryfikacyjna/Wyniki eksperymentów/Wyniki symulacji}


\section[Zakończenie]{Zakończenie/Wnioski/Podsumowanie}


\newpage
\section[Przykłady]{Przykłady rysunków, tabel itp.}
\subsection{Listowanie}
Listuje się w sposób następujący:
\begin{itemize}
	\itemi pierwszy poziom listy - element pierwszy,
	\itemi pierwszy poziom listy - element drugi,
	\begin{itemize}
		\itemii drugi poziom listy - element pierwszy,
		\itemii drugi poziom listy - element drugi,
		\itemii drugi poziom listy - element trzeci,
		\begin{itemize}
			\itemiii trzeci (ostatni) poziom listy - element pierwszy,
			\itemiii trzeci (ostatni) poziom listy - element drugi,
			\itemiii trzeci (ostatni) poziom listy - element trzeci,
		\end{itemize}
		\itemii drugi poziom listy - element czwarty,
		\end{itemize}
	\itemi pierwszy poziom listy - element trzeci,
	\itemi pierwszy poziom listy - element czwarty.
\end{itemize}

\subsection{Cytowania i przypisy}
This document is an example of BibTeX using in bibliography management. Three items 
are cited: \textit{The \LaTeX\ Companion} book \cite{latexcompanion}, the Einstein
journal paper \cite{einstein}, and the Donald Knuth's website \cite{knuthwebsite}. 
The \LaTeX\ related items are \cite{latexcompanion,knuthwebsite}.\footnotemark[1]

\fancyfootnotetextss{1}{Źródło: \url{https://www.sharelatex.com/learn/Bibliography_management_with_bibtex}}{2}{Niestety nie ma automatycznej numeracji przypisów.}{3}{Ich liczba na stronie ograniczona jest do trzech (jednak można to rozszerzyć)}


\newpage
\bibliographystyle{plain} % bibliography style
\bibliography{bibliography} % add bibliography


\end{document}
