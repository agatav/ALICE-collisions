\documentclass[11pt,a4paper,twoside]{article}
\usepackage[includeheadfoot, left=3cm, right=2cm, top=2.5cm, bottom=2.5cm, headheight=19.3pt]{geometry}

\usepackage[polish]{babel}
\usepackage[T1]{fontenc}
\usepackage[utf8]{inputenc}
\usepackage{pslatex}
\addto\captionspolish{\renewcommand{\refname}{Bibliografia}}
\renewcommand{\refname}{Bibliografia}

\let\lll=\relax
\usepackage{amsmath}
\usepackage{amsfonts}
\usepackage{amssymb}
\usepackage{bm}
\usepackage{url}

\usepackage{pdfpages}
\usepackage{footnote}
\usepackage[bookmarks, pagebackref]{hyperref}
\usepackage[none]{hyphenat}
\usepackage{graphicx}
\usepackage{color}
\usepackage{array}
\usepackage{etoolbox, fancyhdr, xcolor}
\usepackage[hang, flushmargin]{footmisc}
\usepackage{setspace}
\usepackage{ragged2e}
\usepackage{MnSymbol}
\usepackage[nottoc]{tocbibind}
\usepackage{titlesec}
\urlstyle{rm}

\usepackage[ampersand]{easylist}
\usepackage{enumitem}
\setlist[itemize]{noitemsep, topsep=0pt, leftmargin=*, itemindent=-1cm}
\usepackage{epsfig}
\usepackage{float}
\usepackage[font={up, footnotesize}, labelfont=bf, singlelinecheck=off, format=hang]{caption}
\usepackage{caption}
\usepackage{subcaption}
\usepackage{listings}
\lstset{language=C++, rulecolor=\color{sapphire}, framerule=1.5pt} % you can change language of your code
\usepackage{colortbl}
\usepackage{tabu}
\usepackage{makecell}
\usepackage{boldline}
\setlength{\arrayrulewidth}{1pt}
\captionsetup[table]{name=Tabela}
\usepackage{multirow}
\usepackage[titletoc, title, header]{appendix}
\definecolor{sapphire}{RGB}{120,150,207}
\definecolor{grafit}{RGB}{60,60,60}

% pdf output setup
\hypersetup{
	unicode,
	pdftoolbar,
	pdfmenubar,
	pdffitwindow,
	pdfstartview = {FitH},
	pdftitle = {Praca Inżynierska}, 
	pdfauthor = {Ahata Valiukevich},
	pdfnewwindow,
	colorlinks,
	linktoc = page,
	% use color sapphire or grafit
	linkcolor = sapphire,
	citecolor = sapphire,
	filecolor = sapphire,
	urlcolor = black
}

\newcommand{\itemi}[1][sapphire]{\item[\color{#1} $\filledsquare$]}
\newcommand{\itemii}[1][sapphire]{\item[\color{#1} $\square$]}
\newcommand{\itemiii}[1][sapphire]{\item[\color{#1} $\bullet$]}

\ListProperties(Hide=100, Progressive=1cm, Style=\color{sapphire}, Style**=\color{black}, Style*=$\square$ ,Style2*=$\bullet$)

\newcommand{\headrulecolor}[1]{\patchcmd{\headrule}{\hrule}{\color{#1}\hrule}{}{}}
\newcommand{\footrulecolor}[1]{\patchcmd{\footrule}{\hrule}{\color{#1}\hrule}{}{}}

\newcommand{\infostyle}[1]
{
	\fancyhf{}
	\fancyhead[LO]{\Large{\textbf{#1}}}

	\renewcommand{\headrulewidth}{1.5pt}
	\headrulecolor{sapphire}
	\justify
	\pagestyle{fancy}
}

\newcommand{\thesisstyle}
{
	\fancyhf{}
	\fancyhead[RO,LE]{\Large{\textbf{\rightmark}}}
	\fancyfoot[LE,RO]{\thepage}

	\renewcommand{\headrulewidth}{1.5pt}
	\renewcommand{\footrulewidth}{1.5pt}
	\headrulecolor{sapphire}
	\footrulecolor{sapphire}
	\justify
	\pagestyle{fancy}
}

\usepackage{etoolbox}
\usepackage{appendix}

\usepackage{tocloft,titlesec}
\newcommand{\listappendixname}{\Large\selectfont{Spis dodatków}}
\newlistof{appendix}{apc}{\listappendixname}

\makeatletter
\AtBeginEnvironment{appendices}{%
  \addcontentsline{toc}{section}{Spis dotatków}
  \write\@auxout{%
    \string\let\string\latex@tf@toc\string\tf@toc% Store the original `\tf@toc` file handle
    \string\let\string\tf@toc\string\tf@apc% 
  }% Naughty trick
    \clearpage
}

\AtEndEnvironment{appendices}{%
  \write\@auxout{%
    \string\let\string\tf@toc\string\latex@tf@toc% 
  }% Naughty trick, restoring the old `\tf@toc` number, to be able to write stuff to the real `.toc` again.
}
\makeatother

\newcommand{\fancyfootnotetext}[2]
{
	\fancypagestyle{footnotes}
	{
    	\fancyfoot[LO,RE]{\parbox{15cm}{\footnotemark[#1]\footnotesize #2}}
	}
	\thispagestyle{footnotes}
}


\newcommand{\fancyfootnotetexts}[4]
{
	\fancypagestyle{footnotes}
	{
    	\fancyfoot[LO,RE]{\parbox{15cm}{\footnotemark[#1]\footnotesize #2 \\ \footnotemark[#3]\footnotesize #4}}
	}
	\thispagestyle{footnotes}
}


\newcommand{\fancyfootnotetextss}[6]
{
	\fancypagestyle{footnotes}
	{
    	\fancyfoot[LO,RE]{\parbox{15cm}{\footnotemark[#1]\footnotesize #2 \\ \footnotemark[#3]\footnotesize #4 \\ \footnotemark[#5]\footnotesize #6 \\}}
	}
	\thispagestyle{footnotes}
}

\titlespacing\section{0pt}{15pt}{10pt} 
\titlespacing\subsection{0pt}{15pt}{10pt} 
\titlespacing\subsubsection{0pt}{15pt}{10pt} 

\setstretch{1.15}
\setlength{\parindent}{0.5cm} 
\setlength{\parskip}{0cm} 

\frenchspacing
\sloppy

\author{} 
\title{} 
\date{}